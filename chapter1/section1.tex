\begin{abstract}
    These are course notes for MAT 5196.
\end{abstract}

\section{January 11, 2022}
\subsection{Introduction}
In this course we will cover important topics of Mathematical Statistics. This course covers methods of hypothesis testing thoery and interval estimation in the context of \textbf{Parametric Statistics} and \textbf{Classical Non parametric Statistics}.

In a typical statistical problem our objective is to get information about the distribution $\vocab{P}$ of a random variable $X$ based on $n$ independent observations $X_1,...,X_n$ of $X$. 

\begin{definition}[Random Sample]
\vocab{Random variable} is defined 
$$
X(S,\PP) \to \SX
$$
Where $S$ is the sample space, and $\PP$ is a probability measure.
\vocab{Random sample} is defined as direct product of 
$$
S \times \cdots \times S \to \underbrace{(\SX \times \cdots \times \SX)}_\text{sample space}
$$
Sample space is just a set of all possible values of a random sample.
\end{definition}
In general for simplicity we assume that $(\SX \times \cdots \times \SX) = \RR^n$

A \vocab{hypothesis} is a statement regarding the parameter of the distribution or distribution itself. The two complementary hypothesis in a hypothesis testing problem are called the \textit{null hypothesis} ($H_0$) and \textit{alternative hypothesis} ($H_1$).
\begin{definition}[Hypothesis test]
    A hypothesis test is a rule that specifies:
    \begin{enumerate}[i]
        \item For which sample values the decision is made to accept $H_0$ as true.
        \item For which sample values $H_0$ is rejected and $H_1$ is accepted as true. 
    \end{enumerate}
\end{definition}
The subset of the sample space for which $H_0$ will be rejected is called the \textbf{critical region} (or rejection region). The complement of the rejection region is called \textbf{acceptance region}.

