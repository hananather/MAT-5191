\section{January 18, 2022}
\subsection{Examples of MP test}
\begin{example}
Let $X_1,..,X_n$ be a random sample from normal $N(\mu, \sigma^2)$ distribution, where $\mu$ is unknown and $\sigma^2$ is known. If we test
$$
H_0: \mu = 0  \text{ vs. } H_1:\mu = \mu_0
$$ for some $\mu_0>0$. The likelihood ratio is equal to 
$$
\frac{f(\boldsymbol{x};\mu_0)}{f(\boldsymbol{x};0} = \frac{exp(\frac{1}{-2 \sigma^2}\sum_{i=1}^{n}(x_i - \mu_0)^2)}{exp(\frac{1}{-2\sigma^2}\sum_{i=1}^{n}x_i^{2})}
$$
From this and Theorem 2.2, the critical region 
$ \left\{ 
\boldsymbol{x}: \frac{f(\boldsymbol{x};\mu_0)}{f(\boldsymbol{x};0} > k
\right \}$ of the most powerful level $\alpha$ test is equivalent to the region 
$ \left\{ 
\boldsymbol{x}: \sum_{i=1}^{n}x_i > k'
\right \}$,
where the constant $k'$ satisfies the level $\alpha$ constraint,
$$
\PP_{H_0} \left( 
\sum_{i=1}^{n}X_i \geq k' = \alpha
\right)
$$
We note that under $H_0$, $\sum_{i=1}^{n}X_i \sim N(0,n\sigma^2)$,
$$
\PP_{H_0} \left(
N(0,1) \geq \frac{k'}{\sigma \sqrt{n}}
\right) = \alpha
$$
or,
$$
1 - \Phi \left( \frac{k'}{\sigma \sqrt{n}} \right ) = \alpha
$$
or,
$$
k' = \sigma \sqrt{n}\Phi^{-1}(1-\alpha).$$ 

Thus, for the observed value
$\boldsymbol{x} = (x_1,...,x_n)$ of $\boldsymbol{X} = (X_1,...,X_n)$, the MP level $\alpha$ test rejects $H_0$ in favour of $H_1$ if 
$$
\sum_{i=1}^{n}x_i > \sigma \sqrt{n} \Phi^{-1}(1- \alpha) \iff  \bar{x} > \frac \sigma{\sqrt{n}}\Phi^{-1}(1-\alpha)
$$

\end{example}

\begin{definition}[p-value]
A p-value $p(\boldsymbol{X})$ is a test statistic satisfying $0 \leq p(\boldsymbol{x}) \leq 1$ for every sample point $\boldsymbol{x}$.  
The \vocab{p-value} or observed size is defined by
$$
p = p(\boldsymbol{X}) = inf \{ \alpha \in (0,1): \boldsymbol{X} \in C_{\alpha} \}
$$
where $C_{\alpha}$ is the critical region of level $\alpha$ test.
\end{definition}