\section{January 13, 2022}

\subsection{Neyman-Pearson Fundamental Lemma}

\begin{itemize}
    \item typically the test $\phi$ that maximizes power against a particular alternative depends on the \textit{alternative}.
    \item there is an important exception: when the alternative is simple, $\Theta = \{ \theta_1 \}$, the problem is completely specified by
    $$
    \underset{\phi}{\text{max }}\beta_{\phi}(\theta) = \underset{\phi}{\text{max }}\EE_{\theta}[\phi(\bX)]
    $$
    subject to the condition
    $$
    \EE_{\theta}[\phi(\bX)] \leq \alpha  \quad \text{for all } \theta \in \Theta_0
    $$
    \item this maximization problem reduces to the mathematical problem of maximixing an integral (or sum) subject to some conditions. The solution to this problem is called the \term{most powerful (MP) test of level alpha}
\end{itemize}


\begin{definition}[Uniformly most powerful test]
    A level $\alpha$ test which maximizes power among all tests of level $\alpha$ is said to be  \vocab{uniformly most powerful (UMP)} level $\alpha$ test. Thus, $\phi$ is UMP level $\alpha$ test if:
    \begin{enumerate}[(i)]
        \item $\underset{\theta \in \Theta_{0}}{\text{sup }} \beta_{\phi}(\theta) = \alpha$
        \item for any other test $\phi^*$ which satisfies (i) has $\beta_{\phi}(\theta) \geq \beta_{\phi^*}(\theta)$  $\forall \theta \in \Theta_{1} $ 
        
    \end{enumerate}
\end{definition}

\begin{theorem}[The Neyman-Pearson Fundamental Lemma]\label{Neyman-Pearson Lemma}
Let $\boldsymbol{X} = (X_1,...,X_n)$ be a random sample from a probability distribution $\boldsymbol{P_{\theta}}$ with pdf/pmf $f(x;\theta)$, $\theta \in \Theta = \{\theta_0, \theta_1\}$. Suppose that we are interested in testing two simple hypothesis 
$
H_0: \theta = \theta_0 \text{ vs. } H_1:\theta = \theta_1
$
at level $\alpha$.
\begin{enumerate}[(a)]
    \item For testing $H_0$ versus $H_1$ there exists a test $\phi$ and a constant $k$ such that 
    \begin{equation}\label{thm2.2.1}
        \EE_{\theta_{0}}\phi(\boldsymbol{X}) = \alpha
    \end{equation}
    and
    \begin{equation}\label{thm2.2.2}
    \phi(\boldsymbol{X}) = 
    \begin{cases}
    1 \quad& \text{if } f( \boldsymbol{x};\theta_1) > kf( \boldsymbol{x}; \theta_0) \\
    0 & \text{if } f( \boldsymbol{x}; \theta_1) < kf( \boldsymbol{x}; \theta_0)
    \end{cases}
    \end{equation}
    \item If a test satisfies \ref{thm2.2.1} and \ref{thm2.2.2} for some $k$, then it is a UMP level $\alpha$ test. 
    \item If $\phi$ is the most powerful at level $alpha$ for testing $H_0$ versus $H_1$, then for some $k$ it satisfies \ref{thm2.2.2}. It also satisfies \ref{thm2.2.1} unless there exists a test of size $< \alpha$ with power 1.
\end{enumerate}
\end{theorem}

\subsection{Geometric Interpretation of Neyman-Pearson Lemma}
If we consider the set
$$
B = \left \{ 
(\alpha,\beta) \in [0,1]^2: \text{there exists a test $\phi$ such that }
\alpha = \EE_{\theta_0}[\phi(\bX)], \beta = \EE_{\theta_1}[\phi(\bX)] 
\right \}
$$
It can be shown that the set $B$ is:
\begin{enumerate}[(a)]
    \item convex;
    \item contains the points $(0,1)$ and $(1,1)$
    \item symmetric about the point $(1/2,1/2)$ in the sense that if $(\alpha,\beta) \in B$ then the point $(1-\alpha,1-\beta)$ also belongs to $B$
    \item closed.
\end{enumerate}